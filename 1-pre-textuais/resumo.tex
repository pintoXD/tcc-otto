

A medição e registro de variáveis como temperatura e umidade de um ambiente é fundamental em locais como armazéns de produtos eletrônicos, estufas agrícolas e bibliotecas. Isso pode ser realizado de maneira manual ou automatizada com o uso de dispositivos \textit{dataloggers}, que além de salvar os dados lidos também podem enviá-los para uma central via \textit{Wi-Fi} ou \textit{Bluetooth}. Porém, os dispositivos \textit{dataloggers} de temperatura e umidade presentes no mercado atualmente e que possuem essa capacidade de comunicação, apresentam um custo muito alto por unidade, podendo inviabilizar sua aquisição por pequenos negócios. Diante disso, propõe-se desenvolver os esquemáticos eletrônicos e leiaute da placa de circuito impresso do \textit{hardware} de um \textit{datalogger} de baixo custo que possa realizar medições de temperatura, umidade e luminosidade. Utilizando como base o módulo microcontrolador ESP32-S3-WROOM1, o \textit{datalogger} pode persistir os dados coletados em um cartão \textit{microSD}, comunicar-se via \textit{Wi-Fi} ou \textit{Bluetooth} e ser alimentado diretamente ou por meio de um conjunto de pilhas AA. 






% Essas medições podem ser persistidas em um cartão micro 






% um \textit{datalogger} que realize medições de umidade, temperatura e luminosidade, que possa persistir os dados em um cartão microSD, se comunicar via Wi-FI ou \textit{Bluetooth} e tenha um baixo custo em relação aos produtos similares no mercado, utilizando como base o módulo microcontrolador ESP32-S3-WROOM1. Para 







% O hardware e o software de um sistema embarcado são desenvolvidos de forma a atenderem especificações e requisitos técnicos específicos para cada aplicação. Com forma de demonstrar isso, propõ-se a criação o hardware de um dispositivo 







% propõe-se a criação do hardware de um dispositivo datalogger de temperatura, umidade relativa e luminosidade, que possua interfaces de comunicação 






% Cada uma dessas partes é desenvolvido de uma maneira particular, mas devido a forte dependência que possuem, existem particularidades que não são cobertos pelas especificações técnicas do projeto mas que devem ser levados em conta para que essas partes funcionem em harmonia.  que utilize como base o módulo microcontrolador ESP32-S3-WROOM1 e que possua um custo final de produção que o torne competitivo frente a produtos semelhantes já existentes no mercado.











% Em Pelas Ondas do Rádio: Cultura Popular, Camponeses e o MEB analisa a participação de camponeses do nordeste brasileiro no Movimento de Educação de Base. A perspectiva da tese é a de demonstrar como os trabalhadores envolvidos com as escolas radiofônicas elaboraram ações para manutenção e reprodução da escola em sua comunidade, visando obter os benefícios necessários à reprodução e melhoria de seu modo de vida. A partir de representações políticas e culturais singulares, dentre as quais vigoraram: um sentido para escola, um papel para o sindicato e para participação política, preceitos do direito de uso da terra e dos direitos do trabalho, assim como, sentidos múltiplos para o uso do rádio como meio de comunicação, informação e lazer, os camponeses do MEB, foram coadjuvantes da proposição católica modernizadora de inícios de 1960. Isto posto, demarca que a ação do camponês nordestino e seu engajamento político, seja no MEB, nos sindicatos rurais, nas Juventudes Agrárias Católicas (JAC’s), no MCP, e nas mais diversas instâncias dos movimentos sociais do período, não se apartaram do processo modernizador. Neste sentido, considera-se que a modernização brasileira foi pauta das instituições, organismos políticos e partidos, assim como, do movimento social, instância em que ela foi ressignificada a partir de elementos da vida material, que envolviam diretamente, no momento em questão, a problemática do direito a terra, do direito a educação e cultura e dos direitos do trabalho.

% Separe as palavras-chave por ponto
\palavraschave{Hardware, Sistema Embarcado, PCI, Esquemático, Leiaute}