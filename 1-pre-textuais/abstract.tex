% In this on the radio waves: popular culture, peasants and the Basic Education Movement we analyze the participation of peasants of the Brazilian northeastern region in the Basic Education Movement. The focus of this thesis is to demonstrate how the labors involved with broadcast schools have elaborated actions for maintaining and spreading the schools in their communities, in order to achieve the necessary means to improve their way of life. Peasants of the Basic Education Movement have been coadjuvant of the modernizing catholic proposition of the early 1960s, by means of quite peculiar political and cultural representations. Some of these representations were: a meaning for the school, a role for the union and for the political participation, precepts of the land use rights and labor rights, and the multiple meanings of the radio as a mass communication, information and leisure medium. This study intends to stress that the actions – and the political enrollment – of the northeastern peasant could not ever be separated from the modernizing process. The connection can be observed in different social movements of the period, such as the Basic Education Movement, rural unions, the Catholic Agrarian Youth and the MCP. In this sense, we consider that, if the Brazilian modernization was a guideline for the institutions, political organisms and parties for the social movement, such a modernization was a guideline of demands based on elements of material life. Those elements included, by that time, the agrarian reform, the educational issue and labor urgencies.

The measurement and recording of variables such as temperature and humidity of an environment is essential in places such as electronics warehouses, agricultural greenhouses and libraries. This can be done manually or automated with the use of dataloggers devices, which in addition to saving the read data can also send it to a central via Wi-Fi or Bluetooth. However, the temperature and humidity dataloggers devices currently available on the market that have this communication capability have a very high cost per unit, which may make it unfeasible for small businesses to acquire them. Therefore, we propose to develop the electronic schematics and PCB layout of the hardware of a low-cost datalogger that can perform temperature, humidity and brightness measurements. Using the ESP32-S3-WROOM1 microcontroller module as a base, the datalogger can persist the collected data on a microSD card, communicate via Wi-Fi or Bluetooth, and be powered either directly or via a set of AA batteries. 

% Separe as Keywords por ponto
\keywords{Hardware, Embedded Systems, Design, PCB, Schematic, Layout}