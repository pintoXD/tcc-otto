\chapter{Conclusões e Trabalhos Futuros}
\label{chap:conclusoes-e-trabalhos-futuros}

Esse trabalho teve por objetivo apresentar o processo de desenvolvimento de um a hardware embarcado, desde sua concepção até sua produção, apresentando as particularidades técnicas envolvidas e fatores externos que podem afetar no projeto final do hardware. 

Contudo, por não ter sido possível produzir a placa para efetuar testes de funcionalidade e desempenho, não foi possível também assim mensurar o consumo energético real do hardware, bem como não também foi possível definir um tempo de autonomia aproximado quando do uso de pilhas.

Assim, para trabalhos futuros pretende-se o seguinte:

\begin{itemize}
    \item Realizar a produção de um número pequeno de placas;
    \item Executar o processo de \textit{bringup} da placa e realizar testes de consumo energético;
    \item Criar um \textit{Board Support Package (BSP)} que implemente drivers, interfaces de programação e rotinas para gerenciamento de consumo do módulo microcontrolador;
    \item Desenvolver um pacote de firmwares que exemplifiquem como usar cada componente presente no hardware.
\end{itemize}





%\label{sec:contribuicoes-do-trabalho}



%\label{sec:limitacoes}







