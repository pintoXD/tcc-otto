\chapter{Fundamentação Teórica}
\label{cap:fundamentacao-teorica}

% Alguns autores preferem fazer uma ``fundamentação teórica'' no segundo capítulo, outros, preferem fazer uma ``revisão da literatura''. Entretanto, isto é particular de cada trabalho e o autor deve escolher o título mais adequado para o capítulo. Consultar o orientador é importante para determinar o título apropriado.

% Evite começar da seção secundária, ou seja, não passe direto do título do capítulo para o título da seção secundária. Escreva um texto para introduzir as seções subsequentes. Lembre-se de utilizar primeira letra maiúscula quando estiver se referindo a um objeto com numeração específica como capítulo, seção, subseção, figura, tabela, quadro, equação, normalmente, se escreve a primeira letra maiúscula da palavra do objeto seguido do \textit{label}. Por exemplo, a Seção \ref{sec:citacoes} explica como fazer citações bibliográficas. Observe no código fonte deste texto como foi feita a referência cruzada. Isso permite enumerar a seção do modo automático o que facilita caso novas seções sejam criadas.  

Neste capítulo será introduzido a definição de sistemas embarcados, bem como alguns de seus tipos, diferenças arquiteturais e quais aplicações se encaixam melhor a cada um desses tipos de sistema. Em seguida, será apresentado a definição de \textit{Dataloggers} e possíveis aplicações





\section{Sistemas Embarcados}\label{sec:definicao_sistemas_embarcados}

% O processamento de informações até o fim dos anos 1980 era normalmente associado aos grandes computadores nos centros de dados, contudo, com a miniaturização dos componentes eletrônicos, isso passou a ser possível também com os computadores pessoais, que são utilizados principalmente para tarefas de escritório,


Sistemas computacionais são normalmente associados a grandes computadores de centros de processamento de dados, computadores de mesa e afins. Contudo, devido a miniaturização de componentes eletrônicos, esses sistemas puderam ser reduzidos ao ponto de fazerem parte da construção de diversos produtos do cotidiano. Impressoras, máquinas registradoras, controles remotos são alguns exemplos desses produtos. Dessa forma, para sistemas computacionais que são parte integrante de um produto ou ferramenta é dado a denominação de sistemas embarcados \cite{vahid2001embedded}.

Para atingir esse objetivo, contudo, um sistema embarcado possui um hardware com limitações de tamanho, consumo energético e poder de processamento para que a aplicação em que é utilizado seja viável. Essas limitações, contudo, fazem com que esse hardware  não possua a mesma padronização que o hardware utilizado em computadores pessoais, o que faz não só com que diversos tipos existam, como também torna difícil realizar um levantamento para se conhecer todos os tipos de componentes que compõem esse tipo de hardware \cite{marwedel2021embedded}. 

Contudo, à partir de características comuns aos sistemas embarcados, é possível definir que ele deve possuir uma estrutura básica que compreende uma unidade de processamento de informações, interfaces de entrada e saída de dados para que possam interagir com o ambiente, memórias para armazenamento de dados, interfaces de comunicação e uma unidade de fornecimento de energia elétrica. 


\subsection{Tecnologias da unidade de processamento}


A unidade de processamento de um sistema embarcado é composta por um dispositivo processador de informações que é responsável por receber e tratar os dados do ambiente de acordo com as especificações da aplicação. Contudo, existem dispositivos que podem ser utilizados sendo eles os seguintes:


\begin{itemize}
    \item \textbf{Processadores de propósito geral: } São dispositivos que podem ser utilizados em diversas aplicações devido sua capacidade de ser programável, possuírem um grande número de instruções disponíveis para  uso e executam pode executar mais de um programa. Com esse tipo de dispositivo, o tempo de de criação e desenvolvimento de um sistema embarcado é menor, mas o custo unitário e o consumo de energia que uma aplicação que o use teria podem ser altos demais;
    
    \item \textbf{Processadores especializados: } São dispositivos que também são programáveis, mas que possuem um conjunto de instruções otimizado para uma determinada classe de aplicações, o que leva a um custo maior de desenvolvimento. Apesar disso,  dependendo da aplicação pode se obter uma boa performance de processamento, consumo e tamanho da aplicação final. Alguns exemplos desse tipo de processadores são o microcontrolador e o \gls{DSP}, enquanto aquele é otimizado para aplicações de controle mas processa poucos dados, esse destinado a aplicações de telefonia ou processamento de áudio por poder processar um grande volume de dados.
    
    
    \item \textbf{Processadores dedicados: } São dispositivos que não podem ser programados e são construídos de forma a atender uma aplicação especifica. Possuem a melhor performance e o menor consumo do gênero, mas necessitam de um grande custo para serem desenvolvido.
\end{itemize}

\subsection{Características Comuns}

Sistemas embarcados, além das limitações de recursos, possuem algumas características que são comuns a todos independentemente da aplicação. A primeira delas é a forma de interação com o ambiente, que devido a falta de suporte a telas e teclados assim como computadores pessoais, que se dá por meio de sensores e atuadores conectados ao hardware do sistema embarcado. Essa forma de interação faz com que esses dispositivos sejam tipicamente sistemas reativos, que são sistemas que estão em constante interação com o ambiente esperando por algum estímulo e executam um conjunto de instruções bem definidas quando são detectados.

Além disso, devido suas características, um sistema embarcado é dedicado somente para uma aplicação, não sendo possível executar instruções de outras aplicações no mesmo dispositivo. Dessa forma, o sistema que controla um forno micro-ondas executa somente as instruções relativas ao controle dos circuitos do forno de acordo com determinadas condições, não sendo possível executar um outro um programa de um jogo, por exemplo. 



% Embora semelhantes, por possuir um propósito bem definido, diferentemente de computadores um sistema embarcado possui um suporte muito menor funcionalidades que não são essenciais o funcionamento da aplicação em que é utilizado, fazendo com que seu hardware possua diversas limitações para que sua aplicação seja viável. 





% Por exemplo, sua unidade de processamento pode ter uma reduzida frequência de operação para economizar energia quando for alimentado por baterias, menos espaço de armazenamento pode estar disponível para que sua fabricação tenha um custo menor, dentre outras características.








% Embora sejam semelhantes, um sistema embarcado difere de um sistema computacional convencional por possuir recursos computacionais limitados e são projetados para uma aplicação específica. Assim, um sistema embarcado responsável pela operação de uma impressora, por exemplo, não terá um grande poder de processamento, não deverá ter um alto consumo energético para não afetar no funcionamento geral do dispositivo e não poderá ser reutilizado em outro dispositivo. Além dessas caraterísticas, um sistema embarcado também precisa atender a restrições como dimensões, peso e custo unitário para que possa ser possível integrá-lo a outros sistemas. 

% Além dessas caraterísticas, a forma de interação com o ambiente também difere em relação aos sistemas convencionais, que possuem a disposição teclados, telas ou afins. Um sistema embarcado por sua vez faz uso de dispositivos sensores e/ou atuadores para receber, processar e responder estímulos do ambiente em que está presente.


% Além disso um sistema embarcado necessita atender determinadas restrições de funcionamento, tais como restrições de consumo


% De forma semelhante a um sistema computacional convencional como computadores pessoais e celulares, um sistema embarcado deve ser capaz de interagir com o ambiente em que está instalado. 

% Sistemas embarcados também precisam ser capazes de receber, processar e responder estímulos do ambientes mas diferentemente de computadores de mesa, celulares ou afins, não podem contar com o uso de teclados e telas para fazer essa interação, utilizando sensores e atuadores para realizar essa tarefa. Dessa forma, 








% \subsection{Características}






% Sistemas embarcados são formados por um conjunto de sensores e atuadores conectados eletronicamente a processadores de informações. Esses processadores, diferentemente dos processadores de sistemas computacionais convencionais, possuem características específicas que permitem dividi-los em processadores de propósito geral, processadores de aplicação específica e processadores dedicados.


% A tecnologia de processadores de uso geral se refere aos dispositivos desenvolvidos para serem utilizados em diversos tipos de aplicações, o que é possibilitado por serem reprogramáveis e possuírem recursos, como um grande número registradores ou uma grande memória de programa, que aceleram a criação e lançamento de um produto no mercado em detrimento de um uso otimizado dos recursos disponíveis. 

% Já os processadores de aplicação específica por sua vez, são dispositivos que assim como os processadores de uso geral, também são reprogramáveis, mas que possuem uma limitação de recursos disponíveis com o objetivo de 



% São comumente utilizados em microcontroladores, circuitos integrados que possuem unidades de processamento, memória e armazenamento em um único \textit{chip}. 

% Já os processadores dedicados são desenvolvidos para atender as necessidades de uma determinada aplicação específica, assim, uma vez que são programados, não é possível alterar alguma de suas funcionalidades. Devido isso, esses dispositivos geralmente possuem um baixo consumo energético e também um baixo valor unitário. 















%%%%%%%%%%%% Sensores e Atuadores %%%%%%%%%%%%%%%%%%
% Sistemas embarcados precisam ser capazes de receber, processar e responder estímulos do ambientes mas diferentemente de computadores de mesa, celulares ou afins, não podem contar com o uso de teclados e telas para fazer essa interação. Dessa forma, um sistema embarcado interage com o ambiente que está inserido por meio do uso de sensores e atuadores. 

\section{Dataloggers}\label{sec:datalogger}

Um \textit{Datalogger} é um dispositivo de funcionamento autônomo, que conta com sensores, memória interna e é capaz de realizar e disponibilizar a leitura de uma ou mais variáveis do ambiente e armazená-lo por um determinado período de tempo que pode ser de dias, meses ou até anos. Contudo, para atender a esse requisito, a taxa de amostragem dessas variáveis é limitada a escala de segundos, do contrário sua memória interna seria ocupada totalmente em um curto período de tempo.


% \section{\textit{Data Acquisition Systems} - DAQs}

\if{0}
    % Esta frase mostra como citar um livro sobre descargas atmosféricas \cite{rakov2003lightning}. Também podem ser citados \textit{sites} como \citeonline{elat2015densidade}. Você precisa escrever o código da referência no arquivo "referencia.bib" dentro da pasta "elementos-pos-textuais". Veja esse, onde estão alguns exemplos que já foram testados.        

    Referenciando outro livro \cite{LangtangenLogg2017}. Texto texto texto texto texto texto texto texto texto texto texto texto texto texto texto texto texto texto texto. Texto texto texto texto texto texto texto texto texto texto texto texto texto texto texto texto texto texto texto. Texto texto texto texto texto texto texto texto texto texto texto texto texto texto texto texto texto texto texto. Texto texto texto texto texto texto texto texto texto texto texto texto texto texto texto texto texto texto texto.

    Referenciando outro site \cite{secretaria1999}. Texto texto texto texto texto texto texto texto texto texto texto texto texto texto texto texto texto texto texto. Texto texto texto texto texto texto texto texto texto texto texto texto texto texto texto texto texto texto texto. Texto texto texto texto texto texto texto texto texto texto texto texto texto texto texto texto texto texto texto. Texto texto texto texto texto texto texto texto texto texto texto texto texto texto texto texto texto texto texto. Citando uma norma \cite{NBR10520:2002}.
        
    Citação de duas referências que concordam entre si \cite{Almeida2018,Gondim2017}. Texto texto texto texto texto texto texto texto texto texto texto texto texto texto texto texto texto texto texto. Texto texto texto texto texto texto texto texto texto texto texto texto texto texto texto texto texto texto texto. Texto texto texto texto texto texto texto texto texto texto texto texto texto texto texto texto texto texto texto. Texto texto texto texto texto texto texto texto texto texto texto texto texto texto texto texto texto texto texto texto texto texto texto texto texto texto. Citando um manual \cite{manuais1989}. 
        
    Outro tipo de citação é a citação literal ou direta com mais de três linhas. Este tipo de citação deve ser destacada com recuo de $4~cm$ da margem esquerda com letra menor (tamanho 10), sem aspas e com espaçamento simples.  Para exemplificar esse tipo de citação, considere a afirmação de \citeonline{feitosa2016}:
    \begin{citacao}
        A cultura é o processo através do qual o homem cria o algo onde antes imperava o nada. Esse algo é toda complexidade de criações simbólicas, de sentidos e significados que damos às coisas e ao mundo. Um ``algo'' que não se sustenta se não se entender os processos culturais como mecanismos de mediação entre nós e os fenômenos. Assim, mais do que apenas um elemento da comunicação, a mediação é, por excelência, cultural. As diversas modalidades de mediação são apenas sotaques diferenciados dessa mediação cultural. Assim é a mediação informacional.
    \end{citacao}
        
    A afirmação do parágrafo anterior também pode ser reproduzida com a citação na final como mostra o exemplo a seguir: 
    \begin{citacao}
        A cultura é o processo através do qual o homem cria o algo onde antes imperava o nada. Esse algo é toda complexidade de criações simbólicas, de sentidos e significados que damos às coisas e ao mundo. Um “algo” que não se sustenta se não se entender os processos culturais como mecanismos de mediação entre nós e os fenômenos. Assim, mais do que apenas um elemento da comunicação, a mediação é, por excelência, cultural. As diversas modalidades de mediação são apenas sotaques diferenciados dessa mediação cultural. Assim é a mediação informacional. \cite{feitosa2016}.
    \end{citacao}
        
%Mais exemplos e opções de citações podem ser encontradas em:
%        https://en.wikibooks.org/wiki/LaTeX/Bibliography_Management
%        https://github.com/cfgnunes/latex-cefetmg/blob/master/latex-cefetmg/03-elementos-pos-textuais/apendices.tex            

\section{Inserindo figuras}\label{sec:figuras}
    
    A Figura \ref{fig:reitoria} apresenta a fotografia da reitoria da Universidade Federal do Ceará. Observe a estrutura do código para ver como inserir figuras. No título, comece especificando o tipo de figura. Por exemplo, fotografia, desenho, diagrama, fluxograma, gráfico e etc. O espaçamento entre linhas no título é de $1~pt$ (espaçamento simples), apenas a primeira letra da frase é maiúscula. As demais palavras são escritas com letra maiúsculas somente quando são nomes próprios e não há ponto final. 
    
    As margens do título da figura são delimitadas pelo tamanho da figura. Por isso, procure ajustar o tamanho da figura para preencher a largura delimitada pelas margens esquerda e direita da página que possui $16~cm$ de largura. Não esqueça de indicar fonte da figura. O autor deve evitar deixar figuras pequenas menores do que $7~cm$ de largura.
    
    A posição da figura deve ser o mais próximo logo após ter sido chamada no texto (a figura nunca deve aparecer antes de ter sido anunciada no texto). 
    
    %troque h pelo b ou t para mudar a posição da figura.
 	\begin{figure}[h!] 
   	    \captionsetup{width=16cm}%Da mesma largura que a figura
		\Caption{\label{fig:reitoria} Fotografia da reitoria da Universidade Federal do Ceará}
		\UFCfig{}{
			\includegraphics[width=16cm]{figuras/exemplo-1}
		}{
			\Fonte{\citeonline{UFC2012}.}
		}	
	\end{figure}
	
    Texto1 texto texto texto texto texto texto texto texto texto texto texto texto texto texto texto texto texto texto texto texto texto texto texto texto texto texto texto texto texto texto texto texto texto texto texto texto texto texto texto texto texto texto texto texto1.

    Texto2 texto texto texto texto texto texto texto texto texto texto texto texto texto texto texto texto texto texto. Texto texto texto texto texto texto texto texto texto texto texto texto texto texto texto texto texto texto texto2.

    Texto3 texto texto texto texto texto texto texto texto texto texto texto texto texto texto texto texto texto texto. Texto texto texto texto texto texto texto texto texto texto texto texto texto texto texto texto texto texto texto3.

    Texto4 texto texto texto texto texto texto texto texto texto texto texto texto texto texto texto texto texto texto. Texto texto texto texto texto texto texto texto texto texto texto texto texto texto texto texto texto texto texto4.

    A Figura \ref{fig:sondas} Texto texto texto texto texto texto texto texto texto texto texto texto texto texto texto texto texto texto texto. Texto texto texto texto texto texto texto texto texto texto texto texto texto texto texto texto texto texto texto3.

	\begin{figure}[h!]
		\centering
		\captionsetup{width=14cm}%Da mesma largura que a figura
		\Caption{\label{fig:sondas} Gráfico da Atmosfera Superior}	
		\UFCfig{}{
			\includegraphics[width=14cm]{figuras/sondas}
		}{
			\Fonte{adaptado da \citeonline{NASA2016}.}}	
	\end{figure}

    Texto5 texto texto texto texto texto texto texto texto texto texto texto texto texto texto texto texto texto texto texto texto texto texto texto texto texto texto texto texto texto texto texto texto texto texto texto texto texto texto texto texto texto texto texto texto5.

    Texto6 texto texto texto texto texto texto texto texto texto texto texto texto texto texto texto texto texto texto texto texto texto texto texto texto texto texto texto texto texto texto texto texto texto texto texto texto texto texto texto texto texto texto texto texto5.

    Texto7 texto texto texto texto texto texto texto texto texto texto texto texto texto texto texto texto texto texto texto texto texto texto texto texto texto texto texto texto texto texto texto texto texto texto texto texto texto texto texto texto texto texto texto texto texto texto texto texto texto texto texto texto texto texto texto texto texto texto texto texto texto texto texto6.

    Evite terminar seções, capítulos e etc com figura. Procure escrever mais.

\section{Inserindo tabelas}\label{sec:tabelas}
    
    A Tabela \ref{tab:exemplo-1}... texto texto texto texto texto texto texto texto texto texto texto texto texto texto texto texto texto texto texto. Texto texto texto texto texto texto texto texto texto texto texto texto texto texto texto texto texto texto texto.
	
	\begin{table}[!h]
	\captionsetup{width=7cm}%Deixe da mesma largura que a tabela
	\Caption{\label{tab:exemplo-1} Um Exemplo de tabela alinhada que pode ser longa ou curta}%
	\IBGEtab{}{%
		\begin{tabular}{ccc}
			\toprule
			Nome & Nascimento & Documento \\
			\midrule \midrule
			Maria da Silva & 11/11/1111 & 111.111.111-11 \\
			Maria da Silva & 11/11/1111 & 111.111.111-11 \\
			Maria da Silva & 11/11/1111 & 111.111.111-11 \\
			\bottomrule
		\end{tabular}%
	}{%
	\Fonte{o autor.}%
	\Nota{esta é uma nota, que diz que os dados são baseados na
		regressão linear.}%
	\Nota[Anotações]{uma anotação adicional, seguida de várias outras.}%
    }
    \end{table}

	%\begin{table}[h!]	
	%	\centering
	%	\Caption{\label{tab:exemplo-1} Exemplo de tabela}	
	%	\UFCtab{}{
	%		\begin{tabular}{cll}
	%			\toprule
	%			Ranking & Exon Coverage & Splice Site Support \\
	%			\midrule \midrule
	%			E1 & Complete coverage by a single transcript & Both splice sites\\
	%			E2 & Complete coverage by more than a single transcript & Both splice sites\\
	%			E3 & Partial coverage & Both splice sites\\
	%			E4 & Partial coverage & One splice site\\
	%			E5 & Complete or partial coverage & No splice sites\\
	%			E6 & No coverage & No splice sites\\
	%			\bottomrule
	%		\end{tabular}
	%	}{
	%	\Fonte{elaborado pelo autor.}
	%}
	%\end{table}

\subsection{Exemplo de subseção} \label{sec:ex_sec}
	
    Texto texto texto texto texto texto texto texto texto texto texto texto texto texto texto texto texto texto texto texto texto texto texto texto texto texto texto texto texto texto texto texto texto texto texto texto texto texto texto texto texto texto texto texto texto.

    %acrlong{DATASUS},\acrlong{DNV},\acrlong{DO},\acrlong{ESF},\acrlong{IBGE},\acrlong{MFC},\acrlong{MI},\acrlong{MS},\acrlong{NV},\acrlong{ODM},\acrlong{OI},\acrlong{OMS},\acrlong{ONU},\acrlong{PNI},\acrlong{PSF},\acrlong{RIPSA},\acrlong{RN},\acrlong{SIM},\acrlong{SINASC},\acrlong{SUS},\acrlong{TMI},\acrlong{TMMFC}


    \begin{alineascomponto}
	    \item Integer non lacinia magna. Aenean tempor lorem tellus, non sodales nisl commodo ut
	    \item Proin mattis placerat risus sit amet laoreet. Praesent sapien arcu, maximus ac fringilla efficitur, vulputate faucibus sem. Donec aliquet velit eros, sit amet elementum dolor pharetra eget
	    \item Integer eget mattis libero. Praesent ex velit, pulvinar at massa vel, fermentum dictum mauris. Ut feugiat accumsan augue, et ultrices ipsum euismod vitae
	    \begin{subalineascomponto}
		    \item Integer non lacinia magna. Aenean tempor lorem tellus, non sodales nisl commodo ut
		    \item Proin mattis placerat risus sit amet laoreet.
	    \end{subalineascomponto}
    \end{alineascomponto}

\subsection{Uso de siglas} \label{sec:siglas}

    Para utilizar siglas, primeiro defina a sigla no arquivo "lista-de-abreviaturas-e-siglas"~ dentro da pasta "1-pre-textuais" com o comando 
    \begin{verbatim}
        \newacronym{ABNT}{ABNT}{Associação Brasileira de Normas Técnicas}
    \end{verbatim}
    Depois chame a sigla com o comando:
    \begin{verbatim}
        \gls{ABNT}
    \end{verbatim}
    Fica assim: \gls{ABNT}. A primeira vez que o comando é usado para uma determinada sigla, aparece o significado por extenso da sigla com a sua abreviação em seguida. A partir da segunda vez que o comando para uma determinada sigla é usado, aparace apenas a sigla. Por exemplo: \gls{ABNT}.  
    
    Veja o código fonte de outros exemplos: Teste de siglas \gls{TEST}, outros exemplos de siglas: \gls{DA}, \gls{MCEG}. 
    Repare que sempre as siglas estão sendo definidas primeiramente no arquivo ``lista-de-abreviaturas-e-siglas''.
\fi